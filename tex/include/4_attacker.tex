% !TEX root = ../tls-cca-privacy.tex
\section{Attacker and Threat Model}\label{sec:attacker}
We define three types of attackers with different motivations and capabilities. 
These attackers are specific points in a possible spectrum of attackers.

Attacker (a) is a powerful entity interested in precisely tracking and identifying users globally.
It is modeled after a nation state or a member of the intelligence community.
This attacker can read network traffic of one or several large Internet backbone networks and/or Internet Exchange Points (IXPs).
In addition, this attacker can typically coerce network operators to provide extended information about users, for example, by mapping IP addresses to network locations.

Attacker (b) is a local or regional entity with access to a single large network.
It could typically be an individual organization or company, with an interest to track roaming and usage patterns of users over different parts of the network.

Attacker (c), the operator of a small network, could also enhance
its tracking capabilities by the use of TLS client certificates, but can already
precisely track typical counts of $<$$10$ users through device names or MAC
addresses. We hence exclude attacker (c) from further analysis.

Currently, attackers (a) or (b) typically will resort to stochastic or opportunistic ``markers'' to identify people.
For example, in the NSA's QuantumInsert program, some markers (``realms'') used are Facebook, HotMail, or YouTube cookies as well as public static IPv4 addresses~\cite[p. 5]{effqi}.
In contrast to the markers mentioned above, we highlight that client certificates used by TLS CCA are cryptographically unique, used by different services, rarely change over time, are wide-spread on billions of devices, and are frequently transmitted, especially when used with mobile devices and always connected services, such as mobile push notification services.

We highlight that attackers (b) and (c) typically also have other possibilities of tracking users within their own networks, based on, for example, MAC addresses or, for cellular operators, IMEI identifiers.
We also emphasize that tracking users based on TLS Client Certificates requires only passive network access, enabling attacks to be conducted (i) in a lightweight manner, and (ii) retrospectively on stored network samples.