% !TEX root = ../tls-cca-privacy.tex
\section{Responsible Disclosure \& Mitigation}%
\label{sec:disclosure}
We disclosed the privacy issues of CCA-based {\apns} authentication to Apple's Product Security Team in October 2016.
Apple's Product Security Team reacted within a day and quickly acknowledged this issue and its severity.
Apple immediately started patch development and held contact with us through several calls to solve this issue and allow us to test the proposed fix.

Despite the complexity of developing and testing this fix in iOS, macOS, iTunes for Windows, and the {\apns} server infrastructure, the patch was already included in the public iOS and macOS security updates released in January 2017.
We tested the patch using the public beta on macOS 10.12.3 Beta 4 (16D30a) and iOS 10.2.1 Beta 4 (14D27).
We find the issue discussed in this paper resolved through a changed usage of TLS 1.2, moving the client authentication into the encrypted Application Layer Data Exchange, as depicted in Figure~\ref{fig:apnsv3}.
The improved version is indicated in the \emph{ClientHello} message with an Application Layer Protocol Negotiation (ALPN) Next Protocol \emph{apns-security-v3} (opposed to \emph{apns-security-v2} for the version with the vulnerability described in this paper).
% !TEX root = ../tls-cca-privacy.tex
\begin{figure}
        \resizebox{\columnwidth}{!}{
                \begin{tikzpicture}
                \node at ([yshift=.1cm] 0,0) [anchor=south,minimum width=1cm] {Client};
                \node at ([yshift=.1cm] 7,0) [anchor=south,minimum width=1cm] {Server};
                
                \draw[-latex,line width=1.4pt] (0,0) to (0,-4.5);
                \draw[-latex,line width=1.4pt] (7,0) to (7,-4.5);
                
                \packet{0,-.5}{7,-.7}{\textcolor{black}{ClientHello}};
                \packet{7,-1.7}{0,-1.9}{
                        \begin{tabular}{c}
                        \textcolor{black}{ServerHello, Certificate,} \\
                        \textcolor{black}{Server Key Exchange}
                        \end{tabular}
                }
                
                \packet{0,-2.4}{7,-2.6}{\textcolor{black}{Client Key Exchange, Change Cipher Spec}};
                \packet{7,-3.1}{0,-3.3}{Change Cipher Spec};
                \dpacket{7,-3.9}{0,-3.9}{\textcolor{TUMBlue}{\underline{\textbf{{[Application Data]}}}}};
                
                \end{tikzpicture}
       } % end resizebox
        \caption{Improved \apns v3 using TLS1.2, postponing CCA into the \textcolor{TUMBlue}{\underline{\textbf{encrypted}}} Application Layer Data Exchange}
        \label{fig:apnsv3}
        \vspace{-1.25em}
\end{figure}

