% !TEX root = ../tls-cca-privacy.tex
\section{Background}\label{sec:background}
This section gives background on TLS and CCA, push notification services in general, and {\apns} and its employment of TLS CCA in particular.
%
\subsection{TLS \& Certificate-Based Authentication}\label{sec:tls}
%
Transport Layer Security (TLS)---often mistakenly still called SSL---is the de facto standard to establish secure communication between systems on today's Internet.
The most noticeable benefit of TLS is that it allows to provide secure communications without modifying or impacting higher-layer protocols.
The latest version 1.2 of TLS is defined in RFC\,5246. %
The first draft of TLS\,1.3~\cite{ietf-tls-rfc5246-bis-16} was published in 2014 but standardization is still work in progress.
TLS provides mechanisms to authenticate both the destination (server) and the initiator (client) of a connection.

The following explanation focuses on authentication and certificate-related aspects of the TLS 1.2 handshake. 
When a client establishes a TLS connection, it first sends a \textit{ClientHello} message, containing the cryptographic ciphers
supported by the client and other information. 
The server responds with a \textit{ServerHello} followed by a \textit{Certificate} message containing one or more certificates to authenticate the server. 
In addition, the server can request a certificate from the client by sending a \textit{CertificateRequest} message. 
This message may include a list of desired Certificate Authorities (CAs) supported for validation. 
The client responds with a \textit{Certificate} message containing the client's certificate chain. 
Only after this unencrypted mutual authentication, both partners establish the desired secure channel. 
A TLS 1.2 mutual authentication is depicted in Figure~\ref{fig:tls12tikz}.

% !TEX root = ../tls-cca-privacy.tex
\begin{figure}
\begin{subfigure}{0.48\textwidth}
\resizebox{\textwidth}{!}{
		\begin{tikzpicture}
\node at ([yshift=.1cm] 0,0) [anchor=south,minimum width=1cm] {Client};
\node at ([yshift=.1cm] 7,0) [anchor=south,minimum width=1cm] {Server};

\draw[-latex,line width=1.4pt] (0,0) to (0,-4.5);
\draw[-latex,line width=1.4pt] (7,0) to (7,-4.5);

% Client Hello is first TLS message , see https://tools.ietf.org/html/rfc5246#section-7.3:
\packet{0,-.5}{7,-.7}{\textcolor{black}{ClientHello}};
\packet{7,-1.7}{0,-1.9}{
		\begin{tabular}{c}
		\textcolor{black}{ServerHello, Certificate,} \\
		\textcolor{black}{\ldots, CertificateRequest, \ldots}
		\end{tabular}
	}

\packet{0,-2.4}{7,-2.6}{\textcolor{black}{Certificate,\ldots , }\textcolor{TUMBlue}{\underline{\textbf{{Finished}}}}};
\packet{7,-3.1}{0,-3.3}{\textcolor{TUMBlue}{\underline{\textbf{{Finished}}}}};
\dpacket{7,-3.9}{0,-3.9}{\textcolor{TUMBlue}{\underline{\textbf{{[Application Data]}}}}};

		\end{tikzpicture}
	} % end resizebox

\caption{TLS 1.2 Handshake}
\label{fig:tls12tikz}
\end{subfigure}
\quad
\begin{subfigure}{0.48\textwidth}
\resizebox{\textwidth}{!}{
	\begin{tikzpicture}

	\node at ([yshift=.1cm] 0,0) [anchor=south,minimum width=1cm] {Client};
	\node at ([yshift=.1cm] 7,0) [anchor=south,minimum width=1cm] {Server};

	\draw[-latex,line width=1.4pt] (0,0) to (0,-4.5);
	\draw[-latex,line width=1.4pt] (7,0)  to (7,-4.5);
	% Details: https://tlswg.github.io/tls13-spec/#rfc.section.2
	\packet{0,-0.5}{7,-0.7}{\textcolor{black}{ClientHello}}
	\packet{7,-2.2}{0,-2.4}{
		\begin{tabular}{c}

		\textcolor{black}{ServerHello},\\ 
		\textcolor{TUMBlue}{\underline{\textbf{\ldots, Certificate, CertificateRequest,}}} \\ 
		\textcolor{TUMBlue}{\underline{\textbf{\ldots, Finished, [Application\,Data]}}} 
		\end{tabular}
	};

	\packet{0,-3.1}{7,-3.3}{\textcolor{TUMBlue}{\underline{\textbf{{Certificate, \ldots, Finished, [Application Data]}}}}} ;
	\dpacket{7,-3.8}{0,-3.8}{\textcolor{TUMBlue}{\underline{\textbf{{[Application Data]}}}}};

	\end{tikzpicture}
}
\caption{TLS 1.3 Handshake}
\label{fig:tls13tikz}
\end{subfigure}\\
\caption{Handshakes in TLS\,1.2 and TLS\,1.3 (draft), highlighting unencrypted and \textcolor{TUMBlue}{\underline{\textbf{encrypted}}} data. Both client and server certificates are encrypted in TLS\,1.3.}
\vspace{-1em}
\end{figure}


For authentication, TLS relies on X.509v3 certificates~(cf. RFC\,5280),
asserting that a cryptographic public key belongs to the certificate's \emph{subject},
making certificates identifiable and attributable to users or devices.
Certificate validation is performed by the system receiving the certificate. 
Validation is performed according to the system's requirements, i.e., the systems do not have to use publicly verifiable certificates issued by well-known certificate authorities but can instead rely on private CA infrastructures.

The current draft for TLS 1.3~\cite{ietf-tls-rfc5246-bis-16} proposes a
different handshake protocol with the specific security goal to protect the
endpoints' identities. First of all, a shared secret is established between
client and server to protect against passive attackers. The client first sends a
\textit{ClientHello} message containing an (EC)DHE key share. The server
responds with a \textit{ServerHello} message containing its key share used to
compute the shared secret. After such a shared secret is established and
communication is encrypted, the server provides the client with its certificate
in the \textit{Certificate} message. Only after sending its certificate, it can
request a client certificate sending a \textit{CertificateRequest} message.
This approach protects both the server and client certificate from observation. 
The proposed handshake for TLS 1.3 is depicted in Figure~\ref{fig:tls13tikz}.
It is important to note that TLS 1.3 can still leak sensitive information since the \textit{ClientHello} (containing the TLS extensions) is still unencrypted. 
These TLS extensions can, for example, leak the target hostname through the \textit{Server Name Indication (SNI)} \textit{ClientHello} extension.


%%%%%%%%%%%%%%%%%%%%%%%%%%%%%%%%%%%%%%%%%

\subsection{Use Cases for Client Certificate Authentication}

TLS in combination with CCA is used in a large
variety of settings, especially when mutual authentication of communication
partners is essential. 
Where passwords are considered insufficient, CCA can provide a level of multi-factor authentication, as recommended by OWASP~\cite{OwaspCCA}.

On network level, CCA can be used with network access control and virtual private
network authentication. One prominent example is enterprise (wireless) network
authentication using 802.1x in combination with the \emph{Extensible Authentication
Protocol (EAP)} (cf. RFC\,5247). EAP is an extensible authentication framework
and supports certificate-based mutual authentication with EAP-TLS (cf. RFC\,5216).
TLS CCA is also used with OpenVPN~\cite{OpenVpn}, one of the most widely used solutions
to create virtual private networks (VPN). OpenVPN uses a custom security
protocol based on shared keys or TLS. In TLS mode, OpenVPN employs X.509
certificates for mutual authentication.

With HTTPS, TLS can be used to authenticate both the web site and the user.
Popular web servers such as Apache, NGINX, and IIS support authentication using CCA.
Websites do not widely use CCA as it is complex for users to install and maintain certificates across multiple devices and browsers.

{\noindent}TLS and CCA are also employed for higher-layer applications such as MQTT (cf. ISO/IEC PRF 20922), a lightweight messaging protocol designed for the ``Internet of Things''. 
MQTT can use TLS and CCA for client authentication, but provides no means to authenticate the server.

\subsection{Push Notification Services}

\textit{Push Notification Services ({\pnses})} are an essential functionality of modern
service ecosystems. 
{\pnses} provide a resource efficient approach for service backends to notify (mobile) devices and applications about
events. 
{\pnses} originate from mobile platforms where resources like energy and network access are limited.
%
All modern mobile platforms are equipped with {\pnses}, often tightly integrated with the operating system: 
Apple's {\apns} on iOS, Firebase Cloud Messaging (FCM) on Google's Android, and Windows Notification Service (WNS) on Windows Phone are prominent examples. 
{\pnses} became the most prominent way to notify applications about service events. 
{\pnses} are also integrated with desktop operating systems and even web browsers (e.g., Google FCM in Chrome) to efficiently notify applications about service events. 
This makes {\pnses} an omnipresent, inevitable link between applications, devices, and service and infrastructure backends. 
Figure~\ref{fig:pstikz} displays this basic concept of {\pnses}.
% !TEX root = ../tls-cca-privacy.tex
%%%%%%%%%%%%%%%%%%%%%%%%%%%%%%%%%%%%%%%%%
\begin{figure}
\begin{scaletikzpicturetowidth}{1\linewidth}
	\begin{tikzpicture}[scale=\tikzscale]
	\pgfdeclarelayer{bg}    % declare background layer
	\pgfsetlayers{bg,main}
	% Push Server
	\draw [fill=white] (4,2) rectangle (6,0) node [pos=0.5,yshift=5] {Push} node [pos=0.5,yshift=-5] {Server};

	% Phone
	\draw [rounded corners] (9.5,0) rectangle (11.2,2) ;
	\draw[rounded corners] (9.6,.4) rectangle (11.1,1.9);
	\draw (10.4,.2)--(10.4,.2);
	\draw[->,dashed](6,1) -- (9.5,1);

	% Servers
	\draw [fill=white] (0,-.5) rectangle (2.25,-1.5) node [pos=0.5] {Server C} ;
		\draw[thick](2.25,-1) -- (4,1);
	\draw [fill=white] (0,.5) rectangle (2.25,1.5) node [pos=0.5] {Server B} ;
		\draw[thick](2.25,1) -- (4,1);
	\draw [fill=white] (0,2.5) rectangle (2.25,3.5) node [pos=0.5] {Server A} ;
		\draw[thick](2.25,3) -- (4,1);

	% Apps
	\draw [fill=white, rounded corners=2mm] (13,-.5) rectangle (15,-1.5) node [pos=0.5] { App C} ;
			\draw[thick](11.2,1) -- (13,-1);
	\draw [fill=white, rounded corners=2mm] (13,.5) rectangle (15,1.5) node [pos=0.5] { App B} ;
			\draw[thick](11.2,1) -- (13,1);
	\draw [fill=white, rounded corners=2mm] (13,2.5) rectangle (15,3.5) node [pos=0.5] { App A} ;
			\draw[thick](11.2,1) -- (13,3);

	\end{tikzpicture}
\end{scaletikzpicturetowidth}
%
\caption{Push Service Architecture: Messages brokered to Apps through the Push Notification Service.}
\label{fig:pstikz}
\end{figure}


When establishing a connection to the {\pns}, both the device and the
third party provider have to authenticate to the service. While for third party
provider, authentication approaches like OAuth are used, device authentication is
not documented for most services due to their proprietary architecture and
tight integration with the operating system. Only Apple documents the use
of TLS and CCA with its security guide~\cite{IosSecurityGuide} and {\apns} documentation~\cite{2016appleapns}.
We investigate communication for Google's FCM and Microsoft's WNS, but do not
find plain-text client certificates. Google's FCM uses TLS 1.2 and TCP ports 5228 through
5230 to connect to its backend. In our analysis for FCM, no
unencrypted client certificates were transmitted with the handshake. The same applies
to Microsoft's WNS, which uses TCP port 443 and TLS 1.2. 

\subsection{{\apnslong} ({\apns})}
\label{sec:apns}
% CONSISTENCY: The current Mac operating system is macOS, originally named "Mac OS X" until 2012 and then "OS X" until 2016.[3] 
%
Within Apple's ecosystem, \apns is a key service for the communication between Apple's service backend and user devices and applications. 
{\apns} was added to mobile devices with iOS 3.0 in 2009, and extended macOS platforms with the release of Mac OS X 10.7 in 2011. 
{\apns} and CCA is also used with \itunes on Windows 7 and later. 
Support for website notifications in Apple's Safari browser was added with the release of Mac OS X 10.9 in 2013.

On its initial activation, each Apple device is provided with a private cryptographic key and a X.509
certificate, stored in the device keychain.
With \itunes, the certificate is created when logging in with
an Apple ID and is stored in Window's certificate store. Mutual authentication
is performed when the device connects to {\apns}: Using TLS 1.2 with CCA as
described in Section~\ref{sec:tls}, first the server's certificate is validated
by the device. Next, the device sends its client certificate, which is validated
by {\apns} to establish a mutually trusted connection~\cite{2016appleapns}.

For \textit{device-to-push-service} communication, {\apns} uses two different TCP
ports~\cite{2016appleapnstrouble}: By default the device tries to connect to an
{\apns} server on TCP port 5223. If TCP port 5223 is not reachable (e.g., due to
port filtering), {\apns} connects to TCP port 443 on WLAN only.
% 
When connected to both cellular and WLAN or wired networks, {\apns} prefers cellular data
links over WLAN or wired connections~\cite{2016appleapnstrouble}. 
We assume this reduces reconnects and provides a stable link for mobile users.

For its backend, {\apns} employs a load balancing architecture, where
devices connect to one of 50 {\apns} DNS records named
\textit{{[1-50]-courier.push.apple.com}}~\cite{2016appleapnstrouble}. 
These names are served by Akamai's DNS service and mapped to
\textit{{[1-50]\-.courier\--push\--apple\-.com\-.akadns\-.net}}
using CNAME records. 
These resolve to a geographically close name, for example, 
\textit{{pop\--eur\--central\--courier\-.push\--apple\-.com\-.akadns\-.net}}
when resolved from Munich. 
These names resolve to a variety of IP addresses in Apple's 17.0.0.0/24 address range.
%
