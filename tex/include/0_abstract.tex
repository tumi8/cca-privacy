% !TEX root = ../tls-cca-privacy.tex
\begin{abstract}%
The design and implementation of cryptographic systems offer many subtle pitfalls.
One such pitfall is that cryptography may create unique identifiers potentially usable to repeatedly and precisely re-identify and hence track users.
This work investigates TLS Client Certificate Authentication (CCA), which currently transmits certificates in plain text.
We demonstrate CCA's impact on client traceability using Apple's {\apnslong} (APNs) as an example. {\apns} is used by all Apple products, employs plain-text CCA, and aims to be constantly connected to its backend.
Its novel combination of large device count, constant connections, device proximity to users and unique client certificates provides for precise client traceability.
We show that passive eavesdropping allows to precisely re-identify and track users and that only ten interception points are required to track more than 80 percent of {\apns} users due to global routing characteristics.
We conduct our work under strong ethical guidelines, responsibly disclose our findings, and can confirm a working patch by Apple for the highlighted issue.
We aim for this work to provide the necessary factual and quantified evidence about negative implications of plain-text CCA to boost deployment of encrypted CCA as in TLS 1.3.
\end{abstract}%