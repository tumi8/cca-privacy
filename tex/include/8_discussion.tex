% !TEX root = ../tls-cca-privacy.tex
\section{Discussion}%
\label{sec:discussion}
%
While negative privacy implications of clear-text certificate transmission in the TLS handshake have been publicly discussed before, the impact of clear-text CCA has never before been empirically quantified against a specific and realistic threat model.
This might have been one of the reasons why remediation of this problem was not a priority in the many years of TLS usage and has been delayed to the major rehaul of the TLS handshake with TLS 1.3.

With this work, we provide evidence that clear-text TLS CCA may have crucial impact on user privacy.
Based on empirical measurements, we show that this impact is easily abusable by the attackers defined in our threat model.
We follow a responsible disclosure approach and find Apple to assess our insights and their implications severe enough to start immediate and expedited patch development for billions of affected devices.

\textbf{Generalization of Results: }%
With our passive measurements taking place at one specific WLAN and wired network, the question arises whether its results generalize to a more heterogeneous network that also includes cellular connections and accompanying Carrier-Grade NATs (CGNs) or other middleboxes.
We strongly argue that our results generalize to those networks.
For geographical tracking of users through passive observations of {\apns} handshakes, the network operator must be able to infer a user location based on externally visible properties of the handshake, typically the source IP address.
We argue that operators of all kinds of networks will have this capability, as it is typically required by law in most countries to resolve abuse and other inquiries.
This capability also enables a global adversary to locate users precisely: By coercing local network operators to pin-point users based on their public IP address, powerful global adversaries can leverage TLS CCA to globally and precisely track users.
Even without such a collaboration CCA-based tracking is useful for an attacker, as often not the precise geographical location is required by the attacker, but a more coarse localization is sufficient, for example, if a user visited a particular country or region. 
Such an approximate localization can still be obtained with approaches such as CGNs in place.
Aside from limited geographical accuracy, an attacker can still learn about user behavior and
infer usage patterns from temporal correlation.

\textbf{Remediation Strategies: }%
We consider the elimination of clear-text TLS CCA from current applications an important vector to enhance privacy in networked systems and discuss several strategies for doing so:
First and foremost, the use of TLS 1.3 is likely the preferred way for most applications. However, TLS 1.3 standardization is not finished to date, and roll-out to a critical mass of devices may take several years.
To mitigate the impact, applications might look to reduce the frequency of TLS CCA submissions by aggressively leveraging TLS session resumption. However, TLS session tickets might by themselves create identifiable patterns.
In a short-term strategy, applications may delay the client certificate submission to the encrypted application layer, but this requires major implementation effort on client and server side.
Therefore, this approach may only be feasible in centralized architectures such as {\apns}.
Also, changes to a username/password approach may be feasible, but usually come with usability downsides.
Another strategy to reduce the number of observable TLS CCA transmissions, as already employed by {\apns}, is the prioritized use of long-lived and stable networks (e..g, cellular links) over typically transient wired and WLAN connections.