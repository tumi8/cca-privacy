% !TEX root = ../tls-cca-privacy.tex
\section{Conclusion and Future Work}\label{sec:summary}
With this work we present---to the best of our knowledge---the first qualitative and
quantitative assessment of CCA privacy implications.
We document the use of TLS Client Certificate Authentication (CCA) by Apple's
Push Notification Service (\apns). As TLS transmits clients certificates in
clear text, the frequent logins of devices at the {\apns} backend provide
opportunity for precise user tracking through highly unique cryptographic
properties of client certificates. We validate our claim by a 17-day passive
capturing at the uplink of a major scientific network where we can spot and
track more than 56,000 {\apns} certificates. We demonstrate how certificate information is suited to
track individual users and derive device information. To display that this
tracking technique would be feasible for a powerful eavesdropper, we show
through global measurements that access to only 10 networks may provide
opportunity to track {\apns} users in over 80\% of access networks, both globally
and nationally.
We highlight that we do not foremost see this as a vulnerability of \apns, but rather a weakness in the TLS 1.2 protocol, and strongly support the
encrypted transmission of client certificates in TLS~1.3.
We hope that the quantification of impact in this study helps to accelerate the adoption of TLS 1.3.

\textbf{Data Release: }
In~\cite{reproduc2017}, we outline our aim for repeatable, replicable and reproducible research as defined by ACM~\cite{AcmArtifacts}.
With ethics-driven exceptions discussed below, we publish all data and source code used to create this publication under \\ 
\centerline{\texttt{\url{https://github.com/tumi8/cca-privacy}}}
This includes code to create data-driven figures, which is also referenced as a clickable hyperlink for each data-driven figure in this work. 
As discussed in Section~\ref{sec:ethical}, we can not provide the passively captured {\apns} certificates for ethical reasons.
However, we publish selected captures that highlight the issues and its solution. 
For figures that build on private data, we provide anonymized datasets.
In addition to our published data, our RIPE Atlas measurements will also be long-term accessible through RIPE Atlas. 

\textbf{Future Work: }%
We plan to further quantify the impact of our research by periodically measuring the amount of clear-text TLS CCA observable, and, specifically, the distribution of fixed {\apns} versions.
Also, the adoption of TLS 1.3 raises interesting and quantifiable questions.
Furthermore, the leakage of sensitive information through unencrypted TLS extensions may be an interesting research field.
Finally, we consider the identification of cryptographically unique identifiers in other authentication protocols an important research goal.

\textbf{Acknowledgments: }
We thank the Leibniz Supercomputing Centre (LRZ) of the Bavarian Academy of
Sciences (BAdW) for their support in conducting our verification. We thank
Apple's Security Team for the good and trustful collaboration, their valuable feedback, and their efforts
to improve security and privacy on the Internet.
This work has been supported by the German Federal Ministry of Education
and Research, project X-CHECK, grant 16KIS0530.
